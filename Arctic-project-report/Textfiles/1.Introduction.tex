\section{Introduction}
The Arctic and the Antarctic are two unique regions on Earth that are characterized by extreme cold temperatures, vast ice sheets, and unique ecosystems. The Arctic is located in the northern hemisphere and is centred around the North Pole.   

This is so \citep{NSIDC_ArcticWeather}.

\subsection{Things to write bout}
\begin{enumerate}
    \item Köppen characteristics
    \item Cryosphere
    \item Geographical features, the atmospheric general circulation, regional weather patterns, and ocean currents
    \item Based on the Intergovernmental Panel on Climate Change (IPCC) Sixth Assessment Report (available online) account briefly for ongoing and future \textbf{climate change} in your study area
    \item human society (ethical aspects)
\end{enumerate}


Sea ice make the sea surface act more like land. In between ocean and land. 

\subsection{Background}
When dealing with the Arctic, it is important to define the region. The Arctic is commonly defined as the area north of the Arctic Circle, which is located at approximately 66\degr33'' N latitude \citep{NSIDC_ArcticWeather}. However, others define the Arctic based on the extent of the Arctic ecosystem, which requires a July mean temperature under 10 \(^\circ C\) \citep{NSIDC_ArcticWeather}. Using this definition, the Arctic can be divided into two main sub-regions, including the Arctic maritime region and the Arctic continental region according to \cite{NSIDC_ArcticWeather}. 

\cite{Serreze_BBarry_ArcticClimateSystem2009}