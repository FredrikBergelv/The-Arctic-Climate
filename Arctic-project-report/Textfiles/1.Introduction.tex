\section{Introduction}
The Arctic and the Antarctic are two unique regions on Earth that are characterized by extreme cold temperatures, vast ice sheets, and unique ecosystems. The Arctic is located in the northern hemisphere and is centred around the North Pole.   


\subsection{Background - The Arctic region}
When dealing with the Arctic, it is important to define the region. The Arctic is commonly defined as the area north of the Arctic Circle, which is located at approximately 66\degr33'' N latitude \citep{NationalSnowandIceDataCenter,serreze2014}. However, others define the Arctic based on the extent of the Arctic ecosystem, which requires a July mean temperature under 10\degr \citep{NationalSnowandIceDataCenter}. Using this definition, the Arctic can be divided into two main sub-regions, including the Arctic maritime region and the Arctic continental region according to \citet{NationalSnowandIceDataCenter}.

The Arctic maritime region includes the Arctic Ocean and its surrounding seas such as the Bering and Greenland Seas, as well as the Labrador Sea and Baffin Bay. This region is sometimes described as a Mediterranean-type ocean due to its limited connection to the Atlantic Ocean and the Pacific Ocean \citep{Danilov2000}. Looking at the bathymetry of the Arctic Ocean, one can observe that the Bering Strait, but also the Canadian Archipelago and the Barents Sea, are shallow with depths of hundreds of meters. Comparing this with the Fram Strait between Svalbard and Greenland, it is much deeper, allowing a bottleneck water flow between the Arctic and the Atlantic \citep{jakobsson2003}. Thus, it is very reasonable to view this ocean as a semi-isolated ocean, just like the Mediterranean.

The Arctic continental region consists of places like Svalbard, northern Canada, Russia, and the Nordic countries. However, the main landmass is Greenland, covering a large area of the Arctic. The Arctic is often split into the Low and High Arctic to distinguish between the more forested regions and the tundra up north \citep{Bliss1997}. The landmass of Greenland is thus characterized by a tundra landscape, although along the northern coast it becomes more suitable to talk about a polar desert due to the lack of moisture \citep{Charlier1969}. Furthermore, only 2--3\% of the polar desert experiences some type of vegetation during the summer months \citep{Bliss1997}.

When describing the Greenlandic landmass, one can use the categorization developed by \citet{Köppen1884} to categorize the terrain. Köppen divided the world’s landmasses into five main climate groups: tropical (A), arid (B), temperate (C), continental (D), and polar (E). Furthermore, each subgroup could be divided into more groups depending on precipitation and later on temperature. Due to the constantly cold nature of the Greenlandic landmass, the region was only divided into two main groups: the tundra (ET) and the ice cap (EF).

The Arctic ice cap is an important concept for understanding the Arctic region due to its high albedo and effect on the region. The ice cover in the Arctic is referred to as the cryosphere. On Greenland, the ice cap covers \SI{1759}{\square\kilo\meter}, which is about 81\% of the total area \citep{serreze2014}. Arctic sea ice also plays a very important role in the region, as it turns the ocean into a land-like state.